\section{Related Work}\label{related_work}
\subsection{Batch Scheduling and Cluster Management}
Classic resource management systems were designed for batch scheduling~\cite{Feit97,FRS04}. Like \pacora, batch scheduling systems can allocate multiple resource types and rely on a gang-scheduling model~\cite{FeRu92}.  However, they tend to use queues and priorities to schedule jobs while trying to keep all resources busy. Resource allocations are always the user's responsibility to specify and pay for. Responsiveness can be improved by buying a higher priority. Few batch systems incorporate deadlines, but one such is \cite{AKKMS95}.

Modern cluster resource management has a similar flavor.  In many systems, such as 
Amazon EC2~\cite{EC2}, Eucalyptus~\cite{eucalyptus}, and Condor~\cite{Condor}, users must specify their resource requirements.  Other systems, such as Hadoop ~\cite{hadoop_fair, hadoop_cap, hadoop_matei} and Quincy~\cite{Quincy} use a fairness policy to assign resources. Mesos~\cite{mesos} uses two-level scheduling to manage resources in a cluster. Mesos decides how many resources to offer to frameworks, and the frameworks decide which resources to accept and how to schedule them. Mesos does not provide a particular resource allocation policy, but is a framework that can support multiple policies. Dominant Resource Fairness~\cite{mesos-DRF}, a generalization of max-min fairness to multiple resources types, has been implemented in Mesos.

\subsection{Model-Based Scheduling and Allocation Frameworks for SLAs and Soft Real-Time Requirements}

There is a growing amount of research in the cloud computing and realtime communities which uses application-specific performance ``models'' to try to schedule to meet deadlines or Service-Level Agreements (SLAs).

In much of this research area\cite{1078472,1078493,1285843,1345325}
the performance models are derived from off-line measurements and are either
interpolations from tables or analytic functions based on queueing theory.
The utility functions typically map the number of servers each execution environment receives
to its performance relative to its requirements, which may be several.
A central arbiter maximizes total utility. The utility functions are not necessarily concave,
so the arbiter must use reinforcement learning or combinatorial search to make allocations.
Each application has a manager that schedules the resources given to it by the arbiter.
%\cite{1078411}
%utility functions
%
Walsh \emph{et al.}\cite{1078411} note the importance of basing utility functions
on the metrics in which QoS is expressed rather than on the raw quantities of resources.
There are other philosophical similarities to \pacora,
but since the objective functions are neither continuous nor convex their optimization is difficult.


%
The two related systems AcOS\cite{AcOS} and Metronome\cite{Metronome} feature hardware-thread based maintenance of ``Heart Rate''
targets using adaptive reinforcement learning techniques.
AcOS also senses thermal conditions and can exploit Dynamic Voltage and Frequency Scaling (DVFS).



A survey of autonomic systems research appears in \cite{1380585}.

In the Redline system\cite{Redline}, compact resource requirement specifications are written by hand to guarantee response times.
Isolation of resources is strong, as in \pacora.  Scheduling is Earliest-Deadline-First.
Admission control is lenient but oversubscription situations are remedied by de-admitting some of the non-interactive applications.


A lot of research focuses measuring resource usage to make co-scheduling decisions.
For example, Calandrino \emph{et al.}\cite{unc} use working set sizes to make co-scheduling decisions and enhance soft real-time behavior. Merkel and Bellosa\cite{merkel-eurosys08} propose \emph{Task Activity Vectors} that describe how much each application uses the various functional units; these vectors are used to balance usage across multiple cores.
%and unbalance usage among hardware threads within each core.
%The intended effect is to distribute chip temperature more evenly, but the idea may be more broadly applicable, \emph{e.g.,} for heterogeneous systems.

Another class of work uses utility functions to maximize system or application performance. Rajkumar \emph{et al.}\cite{828990} propose a system Q-RAM that maximizes the weighted sum of utility functions. Their functions
each of which is a function of the resource allocation to the associated application.
Unlike \pacora, there is no distinction between performance and utility, and
the utility functions are assumed as input rather than being discovered by the system.
They are always nondecreasing in every resource and sometimes are piecewise-linear and concave;
in these cases the optimal allocation is easily found by a form of gradient ascent.
When the utility functions are not concave, a suboptimal greedy algorithm is proposed.

Jockey\cite{Jockey} has some similarities to \pacora: it is intended to handle parallel computation, its utility functions are concave,
and it adapts dynamically to application behavior.
Its performance models are obtained by calibrating either event-based simulation or a version of Amdahl's Law to computations.
Jockey does not optimize total utility but simply increases processors until utility flattens for each application,
\emph{i.e.,} each deadline is met.
A fairly sophisticated control loop prevents oscillatory behavior.




Bodik \emph{et al.}\cite{bodik-acdc09} is also like \pacora in that it builds on-line performance models.
Initially, it uses an \emph{exploration policy} that avoids nearly all SLA violations while it builds the model;
later, it shifts to a controller based on the model it has built.
The models are statistical, and bootstrapping is used to estimate performance variance.
Major changes in the application model are detected and cause model exploration to resume.
The models are not convex or concave in general, and all SLAs must be met with high probability.

Ganapathi \emph{et al.} have had success using machine learning to model application performance and select the best performing configuration in \cite{Archana}.

Guo \emph{et al.}\cite{1331730} point out that much prior work is insufficient for true QoS -- merely partitioning hardware is not enough, because there must also be a way to specify performance targets and an admission control policy for jobs.
%They argue that targets should be expressed in terms of capacity requirements rather than rates.
The framework they present incorporates a scheduler that supports multiple execution modes.



Nesbit \emph{et al.}\cite{1436097} introduce the \emph{Virtual Private Machines} (VPM), a framework for resource allocation and management in multicore systems. A VPM consists of a set of virtual hardware resources, both spatial (physical allocations) and temporal (scheduled time-slices).  They break down the framework components into policies and mechanisms which may be implemented in hardware or software. VPM {\em modeling} maps high-level application objectives into VPM configurations, and VPM {\em translation}, which uses VPM models to assign an acceptable VPM to the application while adhering to system-level policies. A VPM scheduler decides if the system can accommodate all applications. The VPM approach and terminology is similar to \pacora's design at the high-level.  However, Nesbit \emph{et al.} did not perform design, implementation or evaluations of the modeling, translation, or scheduling components suggested in their paper.

%mesh well with our study, which can be seen as a specific implementation of several key aspects of the type of framework they describe (i.e. VPM modeling and translation). 
%
%Unlike traditional virtual machines that only virtualize resource functionality, VPMs virtualize a system's performance and power characteristics, meaning that a VPM has the same performance and power profile as a real machine with an equivalent set of hardware resources.

\subsection{Resource Partitioning and QoS}
\label{sec:rel:pm}


\pacora relies on resource partitioning and Quality-of-Service mechanisms when available to enforce its resource allocation decisions.  Resource partitioning and QoS research is active for on-chip, cluster, and networking resources.  However, the vast majority of research focuses on allocating a single resource type with a fixed policy -- typically fairness.

Some have researched network bandwidth fairness~\cite{Blanquer, Kleinberg99fairnessin, Liu}, while others~\cite{Baruah96proportionateprogress, Baruah_fastscheduling, Zhu} have focused on CPU fairness. 

Hardware partitioning research, which has focused mainly on caches, provides mechanisms to partition research according to a policy designed into the hardware, not the flexible allocations \pacora requires \cite{876484, 967444,1194855,1275005,1194858,1318096,1088154,1399973,1069998,1399982} .  Early work focused on providing adaptive, fair policies that ensure equal performance degradation \cite{605420,1086328}, not guaranteeing responsiveness. More recent proposals have incorporated more sophisticated policy management \cite{1241608,1331730,1152161,1254886}. Iyer\cite{1006246} suggests a priority-based cache QoS framework, CQoS, for shared cache way-partitioning. 
%The priorities might be specified per core, per application, per memory type, or even per memory reference. 
However, simultaneous achievement of performance targets as in \pacora is not addressed.


%In general, these papers focus on designing and proving the effectiveness of particular mechanisms for particular goalswithout a concrete notion of a general-purpose resource allocation or scheduling framework. 
%in which a variety of application-specific QoS requirements can be communicated to an all-purpose resource allocator and scheduler.



%The Singularity operating system\cite{aiken-mspc06} provides process isolation through software rather than hardware.  This isolation in accessibility is not the same as the performance isolation that is desirable for \pacora.

%Other partitioning work has focused on interconnect bandwidth QoS \cite{1382130} or partitioning cache capacity and bandwidth simultaneously \cite{1250671}.

%CQoS: a framework for enabling QoS in shared caches of CMP platforms
%\cite{1006246}

%Suh \emph{et al.}\cite{876484, 967444} and Qureshi and Patt \cite{1194855} monitor individual applications' cache performance to decide partitioning in an attempt to reduce the total amount of cache misses and off-chip memory traffic. 
%A wide variety of proposals exist for multicore last-level cache structures that partition the spatial resources between private and shared data, in an attempt to create a manageable trade-off between capacity for shared data and low latency for private data \cite{1275005,1194858,1318096,1088154,1399973,1069998,1399982}.

%\subsection{Application Performance Modeling}
%
%\fix{add information here}


%Citations supporting PACORA's assumptions. 

%Kumar \emph{et al.}\cite{1006707} demonstrate the performance advantages of heterogeneous cores for mixed workloads using heuristic allocation strategies in both space and time.
%
%Genbrugge and Eeckhout\cite{genbrugge-isca07} demonstrate the importance of adapting to changes in application characteristics, in this case instructions per cycle.

%\todo{I DON'T KNOW WHY \cite{975344,wasserman-book} ARE IN HERE}

%Scheduling threads for constructive cache sharing on CMPs
%\cite{1248396}
%
%Chen \emph{et al.}\cite{1248396} propose parallel depth-first thread scheduling as an alternative to work-stealing for constructive cache sharing.
%The paper discusses the automation of task granularity selection to match cache capacity. This kind of tuning is particularly appropriate for a user-level runtime system.


%\subsection{Autonomic Systems}
%Resource Allocation for Autonomic Data Centers using Analytic Performance Models
%	\cite{1078472}
%	
%Autonomic QoS-aware resource management in grid computing using online performance models
%	\cite{1345325}
%
%On the use of hybrid reinforcement learning for autonomic resource allocation
%\cite{1285843}
%	
%Utility-Function-Driven Resource Allocation in Autonomic Systems
%\cite{1078493}

%Redline: First Class Support for Interactivity in Commodity OSs \cite{Redline}
%Paper presented in OSDI'08.
%

%Jockey: Guaranteed Job Latency in Data Parallel Clusters \cite{Jockey}
%Work from Microsoft about the system they built on top of their Cosomos cluster (Dryad???)

%Discusses how they do the dynamic resource allocation for jobs running in a cluster
%Paper from Eurosys 2012
%


%Automatic Exploration of Datacenter Performance Regimes
%\cite{bodik-acdc09}
%

%AcOS: an Autonomic Management Layer Enhancing Commodity Operating Systems \cite{AcOS}
%Work at the Politecnico di Milano, Italy.
%Master Thesis
%Paper and slides at the CHANGE 2012 Workshop (DAC)
%
%Metronome: OS Level Performance Management via Self Adaptive Computing \cite{Metronome}
%Work related to AcOS, done by people from the Politecnico di Milano (Italy), MIT and Harvard.
%Paper presented in DAC 2012.

%A resource allocation model for QoS management
%\cite{828990}
%

%Utility-Based Cache Partitioning: A Low-Overhead, High-Performance, Runtime Mechanism to Partition Shared Caches
%\cite{1194855}
%

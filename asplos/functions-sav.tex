\section{Application Functions}\label{app_func}
%III.	Application Functions
%	a.	Resource Value Functions
%		i.	Structure
%		ii.	Model Building Approach
%			1.	Who measures the data?
%	b.	Penalty Functions
%		i.	Where do the deadlines come from?
%		ii.	How do we set the slopes?
%	c.	Power Model


In this section, we describe the design of each of \pacora's application-specific functions in more detail.

\subsection*{Response Time Functions}

We choose application responsiveness is highly nonlinear for an increasing variety of applications like streaming media or gaming; 

Unlike penalty functions, which describe user preference, a response time function describes process performance based on its resource assignments.
Response time will commonly vary with time as a process changes phase and makes better or worse use of its resources.
To guarantee the objective function is convex, the response time must be also;
this is a plausible requirement akin to the proverbial ``Law of Diminishing Returns''.

Besides the value of the response time function, its gradient or an approximation to it is useful to estimate the relative response time improvement from each type of resource.  A user-level runtime scheduler that schedules work internal to the process may be a good source of data.
Additionally, the resource manager can allocate a modest amount of a resource and measure the change in response time.
Instead of these, \pacora maintains a parameterized analytic response time model with the partial derivatives evaluated from the model \emph{a priori}.

There are examples of response time versus resource behavior that violate convexity.  One such example sometimes occurs in memory allocation, where “plateaus” can sometimes be seen as in Figure~\ref{f:plat}.
\begin{figure}[b]
\parbox{1.6in}{
\includegraphics*{Plateau1.eps}
\caption{\label{f:plat}Response time function with some resource ``plateaus''.}
}
\hspace{\fill}
\parbox{1.6in}{
\includegraphics*{Plateau2.eps}
\caption{\label{f:plateffect}Net effect of the resource plateaus on the process penalty.}
}
\end{figure}
Such plateaus are typically caused by algorithm adaptations within the process to accommodate variable resource availability.  The response time is really the \emph{minimum} of several convex functions depending on allocation and the point-wise minimum that the process implements fails to preserve convexity.  The effect of the plateaus will be a non-convex penalty as shown in Figure~\ref{f:plateffect} and multiple extrema in the optimization problem will be a likely result.

There are several ways to avoid this problem.  One is based on the observation that such response time functions
will at least be \emph{quasiconvex}.  A function $f$ is quasiconvex if all of its \emph{sublevel sets}
$S_\ell = \{x | f(x) \leq \ell\}$ are convex sets.
Alternatively, $f$ is quasiconvex if its domain is convex and
\begin{displaymath}
f(\theta x + (1-\theta)y) \leq \max(f(x),f(y)), 0 \leq \theta \leq 1
\end{displaymath}

Quasiconvex optimization can be performed by selecting a threshold $\ell$ and replacing the objective function
with a convex constraint function whose sublevel set $S_\ell$ is the same as that of $f$.
Next, one determines whether there is a feasible solution for that particular threshold $\ell$.
Repeated application with a binary search on $\ell$ will reduce the level of feasibility
until the solution is approximated well enough.

Another idea is to use additional constraints to explore convex sub-domains of $\tau$.
For example,the affine constraint $a_{p,r} - \mu \leq 0$ excludes process $p$ from any assignment of resource $r$ exceeding $\mu$.  Similarly, $\mu - a_{p,r} \leq 0$ exludes the opposite possibility.
A binary (or even linear)search of such sub-domains could be used to find the optimal value.

\pacora adopts a simpler idea, modeling response times by functions that are convex by construction
and do not distort response time behavior too much.  This approach is developed more fully below.

\fix{why we picked out model format}
\fix{downsides of convexity}
\fix{model creation ptr}

\subsection*{Penalty Functions}

In tradition systems, responsiveness has been described by a single value (usually called a \emph{priority}) associated with a thread of computation and adjusted within the operating system by a variety of ad-hoc mechanisms.   Priority approaches have no mechanism to understand deadlines or the resources required to meet a deadline and as such must run the highest priority applications as fast as possible on all the resources requested. 

for these real-time applications, performance is measured as sufficient if the deadline is met and insufficient otherwise.

Penalty functions  are generically defined as members of a family of such functions
so that user preferences for a process $p$ (elided in the discussion below)
can be implemented by assigning values to a few well-understood parameters.
As a process grows or diminishes in importance, its penalty function can be modified accordingly.
In a client operating system,the instantaneous management of penalty function modifications
should be highly automated by the system to avoid unduly burdening the user.



To guarantee it is convex and non-decreasing, $s$ must be non-negative.
The response time $\tau$  is of course non-negative,
and it may be sensible (if not strictly necessary) to convene that $d$ is also.
A response time constrained process has a marked change in slope, namely from 0 to $s$, at the point $\tau= d$.
In the most extreme case $s = \infty$ (implying infinite penalty for the system as a whole when $\tau > d$).  ``Softer'' requirements will doubtless be the rule.
For processes without response time constraints one can set $d = 0$.
This defines linear behavior with $s$ as the rate of penalty increase with response time.

The gradient of process penalty with respect to its resource allocations is useful in controlling the optimization algorithm.
By the chain rule, $\partial\pi/\partial a_r = \partial\pi/\partial\tau\cdot\partial\tau/\partial a_r$.
The first term is well-defined but discontinuous at $\tau = d$ with
$\partial\pi/\partial\tau = \mbox{ if } (\tau - d) \leq 0 \mbox{ then } 0 \mbox{ else } s$.
The problem of estimating the partial derivatives $\partial\tau/\partial a_r$ is dealt with below.

\fix{details on how to set these}


\subsection*{Power and Battery Energy}

\section{Problem Convexity}\label{convex}

\subsection*{Penalty Function Convexity}
A few facts about convex functions will be useful in what follows.
First, a \emph{concave} function is one whose negative is convex.
Maximization of a concave function is equivalent to minimization of its convex negative.
An affine function, one whose graph is a straight line in two dimensions or a hyperplane in n dimensions,
is both convex and concave.  A non-negative weighted sum or point-wise maximum (minimum) of convex (concave) functions is convex (concave), as is either kind of function composed with an affine function.  The composition of a convex non-decreasing (concave non-increasing) scalar function with a convex function remains convex (concave).

Each penalty function $\pi$ is the pointwise maximum of two affine functions and is therefore convex.
Moreover, since each penalty function is scalar and nondecreasing,
its composition with a convex response time function will also be convex.

\subsection*{Response Time Convexity}
We now show that response time functions $\tau$ including the various bandwidth amplification functions are convex
in both the bandwidth and memory resources $b_r$ and $m_r$ given any of the possibilities we have considered.
Since norms preserve convexity, this reduces the question to proving each term in the norm is convex.
Since all quantities are positive and both maximum and scaling by a positive constant preserve convexity,
\begin{eqnarray*}
\lefteqn{w/(b\cdot\min(c_1\alpha_1(m),c_2\alpha_2(m)))}   \\
&=& \max(w/(b\cdot c_1\alpha_1(m)),w/(b\cdot c_2\alpha_2(m))).
\end{eqnarray*}
It therefore only remains to show that both $1/\sqrt{b\cdot m}$ and $1/(b\cdot\alpha(m))$ are convex in $b$ and $m$.

A function is defined to be \emph{log-convex} if its logarithm is convex.
A log-convex function is itself convex because exponentiation preserves convexity,
and the product of log-convex functions is convex because the log of the product is the sum of the logs,
each of which is convex by hypothesis.
Now $1/b$ is log-convex for $b > 0$ because $-\log b$ is convex on that domain.
In a similar way, $\log(1/\sqrt{b\cdot m}) = -(\log b + \log m)/2$
and $\log m^{-1/d} = -(\log m)/d$ are convex.
Finally, $\log (1/\log m)$ is convex because its second derivative is positive for $m > 1$:
\begin{eqnarray*}
\frac{d^2}{dm^2}\log (1/\log m) &=& \frac{d^2}{dm^2}(-\log\log m)  \\
                                  &=& \frac{d}{dm}\left(\frac{-1}{m\log m}\right) \\
                                  &=& \frac{1 + \log m}{(m\log m)^2}.
\end{eqnarray*}

Summing up, a response time function for a application might be modeled by the convex function
\begin{eqnarray*}
\tau(w,b,\alpha,m) &=& \sqrt[p]{\sum_j \left(\frac{w_j}{b_j\cdot\alpha_j(m_j)}\right)^p}  \\
                   &=& \|\mbox{diag} wd^T \|_p
\end{eqnarray*}
where the $w_j$ are the parameters of the model (the quantities of work) to be learned,
the components of $d$ satisfy $d_j = 1/(b_j\cdot\alpha_j(m_j))$,
the $b_j$  are the allocations of the bandwidth resources,
the $\alpha_j$ are the bandwidth amplification functions (also to be learned),
the $m_j$ are the allocations of the memory or cache resources that are responsible for the amplifications.
This formulation allows the application response time $\tau$ to be modeled as the $p$-norm of
the component-wise product of a vector $d$ that is computed from the resource allocation
and a learned vector of work quantities $w$.

\subsection*{Handling Quasiconvex Response Time Functions}

There are examples of response time versus resource behavior that violate convexity.  One such example sometimes occurs in memory allocation, where ``plateaus'' can sometimes be seen as in Figure~\ref{f:plat}.

Such plateaus are typically caused by algorithm adaptations within the application to accommodate variable resource availability.  The response time is really the \emph{minimum} of several convex functions depending on allocation and the point-wise minimum that the application implements fails to preserve convexity.  The effect of the plateaus will be a non-convex penalty as shown in Figure~\ref{f:plateffect} and multiple extrema in the optimization problem will be a likely result.

There are several ways to avoid this problem.  One is based on the observation that such response time functions
will at least be \emph{quasiconvex}.  A function $f$ is quasiconvex if all of its \emph{sublevel sets}
$S_\ell = \{x | f(x) \leq \ell\}$ are convex sets.
Alternatively, $f$ is quasiconvex if its domain is convex and
\begin{displaymath}
f(\theta x + (1-\theta)y) \leq \max(f(x),f(y)), 0 \leq \theta \leq 1
\end{displaymath}

Quasiconvex optimization can be performed by selecting a threshold $\ell$ and replacing the objective function
with a convex constraint function whose sublevel set $S_\ell$ is the same as that of $f$.
Next, one determines whether there is a feasible solution for that particular threshold $\ell$.
Repeated application with a binary search on $\ell$ will reduce the level of feasibility
until the solution is approximated well enough.

Another idea is to use additional constraints to explore convex sub-domains of $\tau$.
For example,the affine constraint $a_{p,r} - \mu \leq 0$ excludes application $p$ from any assignment of resource $r$ exceeding $\mu$.  Similarly, $\mu - a_{p,r} \leq 0$ excludes the opposite possibility.
A binary (or even linear)search of such sub-domains could be used to find the optimal value.


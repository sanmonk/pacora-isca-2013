\section{Model Creation}

\subsection*{Response Time Modeling}

While it might be possible to model response times by recording past values and interpolating among them,
this idea has serious shortcomings:
\begin{itemize}
\item The size of the multidimensional response time function tables will be large;
\item Interpolation in many dimensions is computationally expensive;
\item The measurements will be “noisy” and require smoothing;
\item Convexity in the resources may be violated;
\item Gradient estimation will be difficult.
\end{itemize}

An alternative approach is to model the response time functions using parameterized expressions that are convex by construction.
For example, the response time might be modeled as a weighted sum of component terms,
one per bandwidth resource, where each term $w_r/a_r$ is
the amount of work $w_r \geq 0$ divided by $a_r$, the allocation of that bandwidth resource\cite{Snav}.
For example,
one term might model the number of instructions executed divided by total processor MIPS,
another might model storage accesses divided by storage bandwidth allocation and so forth.
Such models will automatically be convex in the allocations because $1/a$ is convex for positive $a$
and because a positively-weighted sum of convex functions is convex.

It is obviously important to guarantee the positivity of the resource allocations.
This can be enforced as the allocations are selected during penalty optimization,
or the response time model can be made to return $\infty$ if any allocation is less than or equal to zero.
This latter idea preserves the convexity of the model and extends its domain to all of $\Re^n$.

Asynchrony and latency tolerance may make response time components overlap partly or fully;
if the latter, then the maximum of the terms might be more appropriate than their sum.
The result will still be convex, though, as will any other norm including the 2-norm,
\emph{i.e.} the square root of the sum of the squares.
This last variation could be viewed as a ``partially overlapped'' compromise between
the 1-norm (sum) describing no overlap and the $\infty$-norm (maximum) describing full overlap.

This scheme also accommodates non-bandwidth resources such as memory,
the general idea being to roughly approximate ``diminishing returns'' in the response time with increasing resources.
For clarity's sake, rather than using $a_r$ indiscriminately for all allocations,
we will denote an allocation of a bandwidth resource by $b_r$ and of a memory resource by $m_r$.

Sometimes a response time component might be better modeled by a term involving a combination of resources.
For example, response time due to memory accesses might be approximated
by a combination of memory bandwidth allocation $b_{r1}$ and cache allocation $m_{r2}$.
Such a model could use the geometric mean of the two allocations in the denominator,
\emph{viz.} $w_{r1,r2}/\sqrt{b_{r1}\cdot m_{r2}}$, without compromising convexity.

This begs the question of how memory affects the response time.
The effect is largely indirect.
Memory permits exploitation of temporal locality and thereby \emph{amplifies} associated bandwidths.
For example, additional main memory may reduce the need for storage or network bandwidth,
and of course increased cache capacity may reduce the need for memory bandwidth.
The effectiveness of cache in reducing bandwidth was studied by
H. T. Kung\cite{Kung}, who developed tight asymptotic bounds on the bandwidth amplification
factor $\alpha(m)$ resulting from a quantity of memory $m$ acting as cache for a variety of computations.
He shows that
\begin{displaymath}
\begin{array}{lll}
\alpha(m) &= \Theta(\sqrt m) & \mbox{for dense linear algebra solvers} \\
          &= \Theta(m^{1/d}) & \mbox{for d-dimensional PDE solvers} \\
          &= \Theta(\log m)  & \mbox{for comparison sorting and FFTs} \\
          &= \Theta(1)       & \mbox{when temporal locality is absent}
\end{array}
\end{displaymath}

For these expressions to make sense, the argument of $\alpha$ should be dimensionless and greater than 1.
Ensuring this might be as simple as letting it be the number of memory resource quanta
(\emph{e.g.} hundreds of memory pages) assigned to the process.
If a process shows diminishing bandwidth amplification as its memory allocation increases, this can be accommodated:
\begin{displaymath}
\alpha(m) = \min(c_1\alpha_1(m),c_2\alpha_2(m)),\;c_1,c_2 \geq 0
\end{displaymath}

Each bandwidth amplification factor might be described by one of the functions above
and included in the denominator of the appropriate component in the response time function model.
For example, the storage response time component for the model of an out-of-core sort process might be
the quantity of storage accesses divided by the product of the storage bandwidth allocation and $\log m$,
the amplification function associated with sorting given a memory allocation of $m$.
Amplification functions for each application might be learned from response time measurements
by observing the effect of varying the associated memory resource while keeping the bandwidth allocation constant.
Alternatively, redundant components, similar except for amplification function, could be included in the model
to let the model fitting process decide among them.

The gradient $\nabla\tau$ is needed by the penalty optimization algorithm.
Since $\tau$ is analytic, generic, and symbolically differentiable
it is a simple matter to compute the gradient of $\tau$ once the model is defined.

\subsection*{On-line Response Time Modeling}

For the moment, assume the model norm $p = 1$ and suppose several response time measurements have already been made using a variety of resource allocations to begin optimizing the application response time.  After enough measurements, discovery of the model parameters $w$ that define the function $\tau$ can be based on a solution to the over-determined linear system
$t=Dw$
where $t$ is a column vector of actual response times measured for the process
and $D$ is a matrix whose ith row $D_{i,*}$ contains the reciprocals of the amplified bandwidth allocations
that generated the corresponding response time measurement $t_i$.
Estimating $w$ is relatively straightforward: a least-squares solution accomplished via
\emph{Q-R factorization}\cite{GoVL} of $D$ will determine the $w$ that minimizes (half the)
square of the \emph residual error $1/2 \|Dw - t\|^2_2 = 1/2 \|\varepsilon\|^2_2$.
The solution proceeds as follows:
\begin{eqnarray*}
t     &=& Dw  - \varepsilon    \\
      &=& QRw - \varepsilon    \\
Q^Tt  &=& Rw  - Q^T\varepsilon
\end{eqnarray*}

It is not always necessary to materialize the orthogonal matrix $Q^T = Q^{-1}$;
the individual elementary orthogonal transformations (Householder reflections or Givens rotations)
that triangularize $R$ by progressively zeroing out partial columns of $D$ can simultaneously be applied to $t$.
The elements of the resulting vector $Q^Tt$ that correspond to zero rows in $R$ comprise $-Q^T\varepsilon$.
Since $Rw$ exactly equals the upper part of $Q^Tt$, the upper part of $Q^T\varepsilon$ is zero. The residual error for the $t_i$
can be found by premultiplying $Q^T\varepsilon$ by $Q$.

Suppose a different model norm $p$ is desired.  If $p = 2$, we might first square each measurement in $t$
and each reciprocal bandwidth term in $D$ and then follow the foregoing procedure.
The elements of the result $w$ will be squares as well, and the 2-norm of the difference in the squared quantities will be minimized.  This is not the same as minimizing the 4-norm; what is being minimized is $1/2\|\mbox{diag}(Dww^TD^T - tt^T)\|^2_2$.

\subsection*{Incremental Least Squares}

As resource allocation continues, more measurements will become available to augment $t$ and $D$.
Moreover, older data may become a poor representation of the current behavior of the process if its characteristics have changed,
presumably as reflected in $Q^T\varepsilon$.
What is needed is a factorization $\tilde{Q}\tilde{R}$ of a new matrix $\tilde{D}$
derived from $D$ by dropping a row, perhaps from the bottom,
and adding a row, perhaps at the top.
Corresponding elements of $t$ are dropped and added to form $\tilde{t}$.

The matrices $\tilde{Q}$ and $\tilde{R}$ can be generated by applying Givens rotations
in the way described in Section 12 of \cite{GoVL} to \emph{downdate} or \emph{update} the factorization
much more cheaply than recomputing it \emph{ab initio}.
The method requires retention and maintenance of $Q^T$ but not of $D$.
Every update in \pacora is preceded by a downdate that makes room for it.
Downdated rows are \emph{not} always the oldest (bottom) ones, but
an update always adds a new top row.
For several reasons, the number of rows $m$ in $R$
will be at least twice the number of columns $n$.
Rows selected for downdating will always be in the lower $m - n$ rows of $R$,
guaranteeing that the most recent $n$ updates are always part of the model.

Downdating makes an instructive example. A row downdate operation applies
a sequence of Givens rotations to the rows of $Q^T$.
The rotations are calculated to set every $Q^T_{i,dd}$, $i \neq dd$ to zero.
In the end only the diagonal element $Q^T_{dd,dd}$ of column $dd$ will be nonzero.
Since $Q^T$ remains orthogonal, the non-diagonal elements of row $dd$ will also have been zeroed automatically
and the diagonal element will have absolute value 1.
These same rotations are concurrently applied to the elements of $Q^T t$ and to the rows of $R$ $(= Q^T D)$
to reflect the effect that these transformations have on $Q^T$.

It is crucial to select pairs of rows and an order of rotations that preserves the upper triangular structure of $R$
while zeroing all but the diagonal entry of the chosen column $dd$ of $Q^T$.
Since $dd$ is always below the diagonal of $R$ it initially will contain only zeros.
It is therefore sufficient to rotate every non-$dd$ row with row $dd$, proceeding from bottom to top.
The first $m - n - 1$ rotations will keep row $R_{dd,*}$ entirely zero,
and the remaining $n$ rotations will introduce nonzeros in $R_{dd,*}$ from right to left.
The effect on $R$ will be to replace zero elements by nonzero elements only within row $dd$.
At this point, except for a possible difference in overall sign, $R_{dd,*} = D_{dd,*}$.

Now the rows from 0 down through $dd$ of the modified matrices $Q^Tt$ and $R$ and both the rows and columns of the modified $Q^T$
are circularly shifted by one position, moving row $dd$ to the top (and column $dd$ of $Q^T$ to the left edge).
The following is the result:
\begin{displaymath}
\begin{array}{lll}
    \left[\begin{array}{cc}
      \pm1  &  0 \\
      0     &  \tilde{Q}^T
   \end{array}\right]
   \left[\begin{array}{c}
      t_{dd} \\
      \tilde{t}
   \end{array}\right]
   &=&
   \left[\begin{array}{c}
      \pm D_{dd,*} \\
      \tilde{R}
   \end{array}\right] w
   \\
   \\
   &-&
   \left[\begin{array}{cc}
      \pm1  &  0 \\
      0     &  \tilde{Q}^T
   \end{array}\right]
   \left[\begin{array}{c}
      \varepsilon_{dd} \\
      \tilde{\varepsilon}
   \end{array}\right]
\end{array}
\end{displaymath}
The top row has thus been decoupled from the rest of the factorization and may either be deleted or updated with new data.

The update process more or less reverses these steps, adding a new top row to $R$ and $t$ and a row and column to $Q^T$.
Then $R$ is made upper triangular once more by a sequence of Givens rotations that zero its sub-diagonal elements
(formerly the diagonal elements of $\tilde{R}$) one at a time.
These rotations are applied not just to $R$ but also to $Q^Tt$ and of course to $Q^T$ itself.

\subsection*{Non-Negativity}

To guarantee convexity of the response time model, the solution $w$ to $t \approx QRw$ must have no negative components.
Intuitively, when a resource is associated with more than a single $w_j$
or when the measured response time increases with allocation then negative $w_j$ may occur.

A requirement for non-negative solutions to least-squares linear algebra problems is common,
so much so that it has a name: \emph{Non Negative Least Squares}, or NNLS.
There are several well-known techniques\cite{ChPl},
but since the method proposed here for online model maintenance calls for
incremental downdates and updates to rows of $Q^T$, $Q^Tt$ and $R$,
the NNLS problem is handled with a complementary scheme
based on the \emph{active-set} method\cite{LaHa} that
also downdates and updates the \emph{columns} of $R$ incrementally,
roughly in the spirit of Algorithm~3 in~\cite{LuDu}.
However, \pacora's algorithm cannot ignore downdated columns of $R$
because subsequent \emph{row} updates and downdates must have due effect
on these columns to allow their later reintroduction via column updates when necessary.

The active-set NNLS method is based on the idea that since the only constraints are variable positivity
then for all components either the variable or its gradient will be zero at a solution point; see~\cite{BoVa}, page~142.
The active set, denoted by \textbf{Z}, comprises the column subscripts $j$ for which the variable $w_j$ is zero and the gradient $v_j$ is positive.
If a column $j$ not currently in \textbf{Z} happens to acquire a negative $w_j$ after a back-solve, $w_j$ is zeroed,
$j$ is moved into \textbf{Z} and column $j$ is downdated in $R$, thereby making the gradient positive.
Conversely, if a column already in \textbf{Z} happens to acquire a negative gradient $v_j$ it is removed from \textbf{Z} and updated in $R$,
allowing it to further reduce the value of the objective function.

The set \textbf{Z} and its complement \textbf{P} are implemented as an index $idx$
containing a single permutation vector of the column subscripts comprising
the sorted members of \textbf{P} followed by the sorted members of \textbf{Z},
and an offset defining the beginning of \textbf{Z} in the vector.
For example, if columns 1, 3, and 4 are in \textbf{Z} and columns 0, 2, and 5 are in \textbf{P}
then the resulting vector is [0 2 5 1 3 4] and the offset is 3.
Since the offset is just the size of the set \textbf{P} it is naturally called $p$.

Columns are left in place in $R$. The columns of $R$ belonging to \textbf{P} are denoted by $R^p$ and those in \textbf{Z} by $R^z$.
The updating or downdating of a column only involves modifying the index $idx$ to redefine \textbf{P} and \textbf{Z} and then
applying Givens rotations to (all of) the rows of $R$ to restore $R^p$ to upper triangular form. 

After initial acquisition of data and $QR$ factorization, each step of \pacora's NNLS algorithm
combines incremental row and column downdates and updates as follows:

\begin{pseudocode}{IncrementalNNLS}{t_0,d_0}
\LOCAL{R,Q^T,Q^Tt,w,v,idx,d,u,done}                              \\
R,Q^T,Q^Tt \GETS \textsc{DndtRow}(R,Q^T,Q^Tt,idx)           \\
R,Q^T,Q^Tt \GETS \textsc{UpdtRow}(t_0,d_0,R,Q^T,Q^Tt,idx)     \\
w \GETS \textsc{BackSolve}(R,Q^Tt,idx)                          \\
v \GETS \textsc{Gradient}(R,Q^Tt,idx)                    \\
\REPEAT
  done \GETS \TRUE                                              \\  
  d \GETS \arg\min(w)                                          \\
  \IF w_d < 0 \THEN                                            \\
  \BEGIN
    done \GETS \FALSE                                         \\
    R,Q^T,Q^Tt,idx \GETS \textsc{DndtCol}(R,Q^T,Q^Tt,idx,d)   \\
    w \GETS \textsc{BackSolve}(R,Q^Tt,idx)                    \\
    v \GETS \textsc{Gradient}(R,Q^Tt,idx)              \\
  \END                                                        \\
  u \GETS \arg\min(v)                                         \\
  \IF v_u < 0 \THEN                                           \\
  \BEGIN
    done \GETS \FALSE                                         \\
    R,Q^T,Q^Tt,idx \GETS \textsc{UpdtCol}(R,Q^T,Q^Tt,idx,u)     \\
    w \GETS \textsc{BackSolve}(R,Q^Tt,idx)                    \\
    v \GETS \textsc{Gradient}(R,Q^Tt,idx)              \\
  \END                                                        \\
\UNTIL done                                                   \\ 
\RETURN{w,v}                                                  \\     
\end{pseudocode}
 
When a column indexed by $d$ in $R^p$ is downdated because $w_d < 0$, that column is moved from $p$ to $z$ in the index.
To restore $R^p$ to upper triangular form, Givens rotations are applied to $R$ at rows $R_{d,*}$ and $R_{k,*}$ 
where $d < k < p$. The row subscripts $k$ are used in decreasing order from $p-1$ down to $d+1$,
and each rotation zeros the subdiagonal element of the column indexed by $k$. 
As usual, these rotations are also applied to $Q^T$ and $Q^Tt$.
The result in $R^z$ is a ``spike'' of nonzeros in the column that was moved;
it can eventually extend to the bottom of $R$ as \emph{row} updates occur.

Column movements from \textbf{Z} to \textbf{P} are based on the gradient $v$ of the objective function, namely
\begin{eqnarray*}
v &=& 1/2\nabla\|Dw - t\|^2_2 \\
  &=& D^T(Dw - t)             \\
  &=& R^TQ^T(QRw - t)         \\
  &=& R^T(Rw - Q^Tt)          \\
  &=& R^T(-Q^T\varepsilon).
\end{eqnarray*}
If for some column in \textbf{Z} the inner product of the corresponding spiked row in $R^T$ and $-Q^T\varepsilon$ is negative,
the column subscript must be moved to \textsc{P}.
Updating $R^p$ reverses the downdating steps by zeroing the spike via a sequence of Givens rotations on $R$
between adjacent pairs of rows, starting at the bottom and ending at $m,m+1$ where $m$ is the new index of the column.
These rotations conveniently extend the columns $k$ with $m < k < p$ by one,
thus restoring $R^p$ to upper triangular form. Once again, the rotations are also applied to $Q^T$ and $Q^Tt$.

A new gradient computation and new back-solve for $w$ are clearly necessary after either downdates or updates to columns of $R$.

\subsection*{Model Rank Preservation}

If care is not taken in the allocation process,
the rows of $R$ may become linearly dependent
to such an extent that its rank is insufficient to determine $w$.
This might be the result of repetitions in resource assignment updates.
There are several possible ways to avoid this \emph{rank-deficiency} problem.
The characteristics of $R$ depend on both the resource optimization trajectory and the
choices made in the downdate-update algorithm.
In particular, deciding whether to downdate the bottom row of $R$ or some ``younger'' row
will depend on whether the result would become rank-deficient.
This approach decouples allocation optimization from performance model maintenance
and places responsibility upon the latter to always keep enough history to determine a $w$.

Deciding in advance whether downdating a row of $R$ will reduce its rank
is equivalent to predicting whether one of the Givens rotations, when applied to $R$,
will zero or nearly zero a diagonal entry of $R$.
This is particularly easy to discover because $dd$, the row to be downdated, is initially all zeros in $R$,
\emph{i.e.} in the lower part of the matrix.
In this situation a diagonal entry of $R$, $R_{i,i}$ say, will be compromised if and only if the
cosine of the Givens rotation that involves rows $dd$ and $i$ is nearly zero.
The result will be an interchange of the zero in $R_{dd,i}$ with the nonzero diagonal element $R_{i,i}$.
$R_{dd,i}$ is zero before the rotation because
$R$ was originally upper triangular and prior rotations only involved row subscripts greater than $i$.

\pacora keeps track of the sequence of values in $Q^T_{dd,dd}$ without actually changing $Q^T$
so that if the downdate at location $dd$ is eventually aborted there is nothing to undo.
It is also possible to remember the sines and cosines of the sequence of rotations
so they don't have to be recomputed if success ensues.
A rank-preserving row to downdate will always be available as long as $R$ is sufficiently ``tall''.
Having at least twice as many rows as columns is enough since the number of available rows to downdate
matches or exceeds the maximum possible rank of $R$.

\subsection*{Outliers and Phase Changes}

Some response time measurements may be ``noisy'' or even erroneous.
A weakness of least-squares modeling is the high importance it gives to outlying values.
On the other hand, when an application changes phase it is important to adapt quickly,
and what looks like an outlier when it first appears may be a harbinger of change.
What is needed is a way to discard either old or outlying data
with a judicious balance between age and anomaly.

The downdating algorithm accomplishes this by weighting the errors in $\varepsilon = Q(Q^Tt - Rw)$
between the predicted response times $\tau$ and the measured ones $t$ by a factor
that increases exponentially with the age $g(i)$ of the $i$th error $\varepsilon_i$.
Age can be modeled coarsely by the number of time quanta of some size since the measurement;
\pacora simply lets $g(i) = i$.
The weighting factor for the $i$th row is then $\eta^{g(i)}$ where $\eta$ is a constant somewhat greater than 1.
The candidate row to downdate is the row with the largest weighted error, \emph{i.e.}
\begin{displaymath}
dd = \arg\max_i |\varepsilon_i| \cdot \eta^{g(i)}
\end{displaymath}






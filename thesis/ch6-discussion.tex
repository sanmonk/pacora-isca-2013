\chapter{Discussion}\label{discuss}

There are two main sources of challenges for \pacora's design: performance non-convexity and performance variability.  In the following section we describe possible techniques for coping with these challenges.

The main concern with performance non-convexity and variability is their effects on the accuracy of the response time functions.  However, an important result we have found while evaluating \pacora is that model accuracy has less impact on the quality of resource allocation decisions than we anticipated.  When experimenting with possible models for the RTFs, we found that while some models were always a little too inaccurate and did degrade the performance of the resource allocation decisions, once a model crossed a certain threshold of accuracy then better models provided insignificant improvement in resource allocation decisions.  Although there is a slight correlation between model accuracy and decision quality, many decisions with inaccurate models still result in near optimal allocations.  This effect enables \pacora's model-based design to be feasible in a noisy system with real applications.

\section{Outliers and Performance Non-Convexity}

Outliers and performance non-convexity can be handled with a combination of two techniques.  Outliers should be thrown out during the model creation phase as described in Section~\ref{RTFs} to prevent those points from distorting the accuracy of the other allocations.  The second part of the solution is to keep track of these points with extreme error in the model and use heuristics to adjust the resource allocations to avoid such points.  We did not implement this second idea in our current evaluation, but it is a subject of future study.  In our study, we experienced outliers for a few applications with a particular number of hyperthreads and for many applications when given only one cache way. However, as responsiveness, predictability, and efficiency increase in importance for systems, we expect to see an increased number of chip designs that provide more performance convexity.

\section{Variability}

We observe three main sources of variability in application performance: phase changes, performance changes due to differing inputs, and variable resource performance due to external causes or interference from sharing.

\section*{Phases}
As described in Section~\ref{RTFs}, phases can be handled with online creation of the RTFs.  However, another option is to build different models, one per phase, and swap models when a change occurs.  We used this method for the video application in Section~\ref{eval-dynamic}.  This approach has the advantage that it can more rapidly adapt to phases; however it requires identifying phases and additional space to store the extra models. Phase detection is an active area of research, and \cite{dhodapkar-micro03} provides an overview of techniques.  Another possible approach is to build a model that represents the resource requirements of the most demanding phase.  The system can be designed make use of the idle resources when available or power management mechanisms can put them in low-power mode.

\section*{Input Dependence}
Some applications may significantly change performance as a function of their inputs. In the case of our video application we ignored its input dependence without significant effect.  However for other applications the effect may be more pronounced. As with phases, a solution might be to keep multiple models for the application and select one based on the current input.  A different solution is to make the RTFs \emph{stochastic models} representing the distribution of response times of the application.  Stochastic models for \pacora are discussed in more detail in~\cite{pacora_tr}.

\section*{Resource Variability}
In our study we experience resource variability both in the form of non-deterministic resources such as the network connection in \texttt{ tradebeans} or dynamic frequency changes in the CPU and shared resources such as memory bandwidth. Stochastic models are most likely the right way to deal with non-deterministic resources and may be particularly necessary for representing disk-based storage in warehouse scale computing.

Shared resources can also be handled with stochastic models.  However, since the models can be built online while other applications are running, the interference from a loaded machine is already captured in the model.  In our evaluation of static allocation, we built the models in isolation but found that \pacora was still able to make near-optimal resource allocations for a loaded machine despite the shared resources.

While there will likely always be some shared resources, we have found
a trend towards minimizing interference in emerging chip designs, as
efficiency and predictability begin to trump utilization as primary
concerns.


%other types of non-deterministic apps?



% don't forget cpu frequency







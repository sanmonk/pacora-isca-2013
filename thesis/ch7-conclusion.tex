\chapter{Conclusion and Future Work}
\pacora takes a different approach to resource allocation than traditional systems, relying heavily on application-specific functions built through measurement and convex optimization.  Using convex optimization lets \pacora perform real-time resource allocation inexpensively, enabling \pacora to dynamically allocate resources to adjust to the changing state of the the system.  \pacora makes resource allocation decisions in less than \SI{350}{\micro\second} in the worst case and often faster than \SI{50}{\micro\second} in our manycore OS implementation. By building application-specific functions online and formulating resource allocation as an optimization problem, \pacora is able to accomplish multi-dimensional resource allocation on a general set of resources, thereby handling heterogeneity and the growing diversity of modern hardware while protecting application developers from needing to understand resources. Allocation decisions are near optimal---only 2\% from the best possible allocation on average, and \pacora only requires a few hundred bytes of additional storage per application.
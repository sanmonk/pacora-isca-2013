\section{Introduction}


The demands on modern systems have changed.  Users demand responsive applications often containing high-quality multimedia that requires real-time guarantees.  Meeting this responsiveness goal is a challenge for all types of systems including cloud systems, databases, webservers, client operating systems, and emerging distributed embedded systems. Additionally battery life and system power are extremely important, thereby forcing systems to try to find more efficient ways to meet the quality-of-service demands of their workloads.

Ideally, applications, jobs, or queries with strict performance requirements should be given just enough system resources (\emph{e.g.,} nodes, processor cores, cache slices, memory pages, various kinds of bandwidth) to meet these requirements consistently, without unnecessarily siphoning resources from other applications. However, executing multiple parallel, real-time applications while satisfying  \emph{Quality-of-Service} (QoS) requirements is a complex optimization problem, particularly as modern hardware diversifies to include a variety of parallel architectures (\emph{e.g.,} multicore, gpus).  Historically operating systems have not provided useful mechanisms that implement stronger performance guarantees and resource allocation has been rather unsystematic, making it difficult to reason about the expected response time of an application. 

Consequently, predictability has traditionally been obtained at a significant expense by designing for the worst case and over-provisioning.  Evidence of this behavior can be found in current systems of all sizes.  OSs describe responsiveness with a single value (usually called a \emph{priority}) associated with a thread of computation and adjusted within the operating system by a variety of ad-hoc mechanisms. Other shared resources either employ independent machinery (\emph{e.g.,} memory, caches), or are deemed so abundant as to require no explicit management at all (\emph{e.g.,} I/O, network bandwidth).
 Priority approaches have no mechanism to understand deadlines or the resources required to meet a deadline and as such must run the highest priority applications as fast as possible on all the resources requested.   As a result, interactive and realtime applications are often run needlessly fast with significantly over-provisioned resources --wasting power and energy and preventing other applications from using the resources.  

Some mobile systems have gone so far as to limit which applications can run in the background~\cite{iOsDev} in order to preserve responsiveness and battery life, despite the obvious concerns this raises for user experience.  Cloud computing providers routinely utilize their clusters at only 10\% to 50\% to keep the system responsive despite the additional operational costs of consuming electricity and the significant impact to the capital costs of the infrastructure~\cite{Barroso2009,Hennessy2011}.   In some cases, clusters only run a single application on each cluster to avoid unexpected interference.  Similarly the realtime community has used completely separate systems for each application to provide QoS despite the high-cost of specialization and low utilization with this approach.

In this paper, we present \pacora, a resource allocation framework, which is designed to provide responsiveness guarantees to a simultaneous mix of high-throughput parallel, interactive, and real-time applications in an efficient, scalable manner.  Unlike traditional systems, in order to provide predictable responsive times \pacora considers all resource types when making decisions--providing the complete set of resources an application will need to meet its deadline.   \pacora builds application-specific models of applications through measurement to determine the resources required to meet the deadline and leverages convex optimization with these application performance models to determine the optimal amount of each resource to give each application, enabling the system to make trade-offs between application QoS/responsiveness, system performance, and energy efficiency. 

We believe \pacora is applicable to many resource allocation scenarios including cloud providers determining how much to give each job to avoid violating SLAs, databases allocating resources to queries, and distributed embedded systems allocating bandwidth between devices and sensors.  In this paper we choose to study client applications and implement \pacora in a general-purpose operating system because we believe this scenario has some of the most significant resource allocation challenges: the constantly changing environment requires low overhead and fast response times from \pacora;  shared resources create more interference between applications; and the applications are more likely to be written by domain experts, thus less highly optimized.

\fix{Add Performance Numbers}


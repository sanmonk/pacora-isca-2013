\section{\pacora Architecture}\label{sys_design}

%II.	System Architecture
%		i.	Application Specific
%		ii.	Resources Matter
%		iii.	Measurement
%		iv.	Optimization
%	a.	Convex by construction
%		i.	Fairly representative of the actual system
%			1.	(Systems should be more convex)
%		ii.	Low Overhead
%	b.	Explanation of resources
%	c.	Performance Models Built Online
%		i.	Explanation of performance
%		ii.	When does it run?
%	d.	Penalty functions
%	e.	Dynamic Optimization
%		i.	When does it run?


%\pacora, which stands for Performance-Aware Convex Optimization for Resource Allocation, is a framework designed to determine the proper amount of each resource type to assign to each application.  For our purposes an application is the entity to which the operating system allocates resources: it can viewed as a complete application (\emph{e.g.,} a video player) or a component of an application (\emph{e.g.,} a music synthesizer) or a process (\emph{e.g.,} indexing).

\pacora is a framework designed to determine the proper amount of each
resource type to give each application.  For example, consider a video
conference scenario where each participant requires a separate,
performance-guaranteed video stream.  New participants may join the
conference and others may leave, increasing or decreasing the number
of streams running at any given time.  Simultaneously, participants
may be collaborating through web browsers, or watching shared video
clips and web searching, while their systems run compute-intensive
background tasks such as updates, virus scans, or file indexing.
Although it may be relatively straightforward to provide
responsiveness guarantees for individual applications such as video
streams, the real challenge is to do so without reserving excessive
resources, which will compromise system utilization or power
consumption. The purpose of \pacora is to dynamically assign resources
across multiple applications to guarantee responsiveness without
over-provisioning and to adapt allocations as the application mix
changes.

For our purposes an application is an entity to which the system
allocates resources: these can be a complete application (\emph{e.g.,}
a video player), a component of an application (\emph{e.g.,} a music
synthesizer), a background OS process (\emph{e.g.,} indexing), a job
in warehouse-scale computing, or a distributed application in a
distributed embedded system.

\pacora is designed for systems where resource allocation is separated
from scheduling.  This split enables the use of application-specific
scheduling policies, which have the potential to be easier to design
and more efficient than general-purpose schedulers that have to work
for everything.  This approach leaves the system to focus on the
problem of \emph{how many} of each resource type to assign to each
application.  In client machines, \pacora is used to make coarse-grain
resource-allocation decisions (\emph{e.g.,} cores and memory pages) at
the OS level, while the micro-management of these resources to run
application tasks is left to user-level runtimes such as Intel
Threaded Building Blocks~\cite{CoMa08} or Lithe~\cite{lithe}, and to
user-level memory managers.  In a cloud environment, \pacora allocates
resources (\emph{e.g.,} nodes and storage) to jobs, and scheduling is
left to other entities such as the MapReduce framework\cite{mapreduce}
or the node OS.  \pacora can also be used in a hypervisor to allocate
resources among guest OSes.

\pacora formulates resource allocation as an optimization problem
built from two types of application-specific functions: a
response-time function and a penalty function. The response-time
function represents the performance of the application with different
resources, and is built with runtime measurements.  The penalty
function represents the user-level goals for the application
(\emph{i.e.} the deadline and how important it is to make that
deadline). \pacora then uses convex optimization\cite{BoVa} to
determine the ideal resource allocation across all active
applications.  The following subsections briefly introduce the primary
components of \pacora and the optimization problem itself.

\subsection{Resources}

In our client system, resources are anything that the system can
``partition'' in hardware or software: specifically we use cores,
network bandwidth, cache ways, and memory pages in our system.
However, other resources could be easily added assuming they have QoS
enforcement mechanisms.  The other scenarios would have resources that
perform similar functions (compute, network, capacity), but at a
different scale. For warehouse-scale computing, resources are more
likely to be different types of nodes, network bandwidth, and
storage. For distributed embedded systems, resources would include
compute devices, link bandwidths, and memories.

\subsection{Response-Time Functions}

Response-time functions (RTF) represent the expected \emph{response
  time} of an application as a function of the resources allocated to
the application. The response time is an application-specific measure
of the performance of the application.  For example, the response time
of an application might be:
    \begin{itemize}\itemsep0pt \parskip0pt \parsep5pt
    \item The time from a mouse click to its result;
    \item The time to produce a frame;
    \item The time from a service request to its response;
    \item The time from job launch to job completion;
    \item The time to execute a specified amount of work.
    \end{itemize}

The RTFs are built to be convex functions.  All applications have a
function of the same form but the application-specific weights are set
using the performance history of the application.  RTFs are designed
to capture information such as how well an application scales with a
particular resource. As a result, RTFs naturally support
heterogeneity.  Each CPU or GPU type is simply viewed as a different
resource type by the system, and thus the RTFs will represent how
effectively an application uses a particular type of
core. Figure~\ref{sample_rtf} shows two example RTFs we have created
from applications we studied.

\begin{figure}[hb]
\includegraphics*[bb=0 0 360 360,width=.49\columnwidth]{bfs-fig.pdf}
\includegraphics*[bb=0 0 360
  360,width=.49\columnwidth]{streamcluster-fig.pdf}
\caption{\label{sample_rtf} Response Time Functions for a
  breadth-first search algorithm and streamcluster from PARSEC. We show two resource dimensions: cores and cache slices.}
\end{figure}

Equation~\ref{rtf_eq} below shows the form of the RTF we use in \pacora.

\begin{equation}\label{rtf_eq}
\tau(w,a) = \tau_0 + \sum_{i\in n,j\in n}{\frac{w_{i,j}}{\sqrt{a_i * a_j}}}
\end{equation}

Here $\tau$ represents the response time, $i$ and $j$ represent
resource types, $n$ represents the total number of resource types,
$a_{i}$ and $a_{j}$ represent the allocation of resources types $i$
and $j$, and $w_{i,j}$ represents the application-specific weight for
the term representing resources $i$ resource $j$.  The dimensionality
of the function increases linearly with the number of resource
types. The RTFs will be described further in Section~\ref{app_func}.

\pacora assumes some amount of performance isolation between
applications.  In order for the RTFs to accurately reflect the
expected response times of the applications, it is important the
response time does not change much as a function of the other
applications currently running on the machine.  However, the
performance isolation need not be completely perfect: all of our
evaluation was run on current x86 hardware with some shared resources,
and \pacora still performs quite well. Section~\ref{discuss} discusses
handling shared resources in more detail.

\subsection{Penalty Functions}
Penalty functions are designed to represent the user-level goals of
the application. They are similar to priorities but are functions of
the response time rather than simply values, so they can explicitly
represent deadlines.  Knowing the deadlines lets the system make
optimizations that are difficult in today's systems, such as running
just fast enough to make the deadline. Like priorities, the penalty
functions are set by the system on behalf of the user.

\pacora's penalty function $\pi$ is a non-decreasing piecewise-linear
function of the response time $\tau$ with the form:

\begin{equation}\label{pen_eq}
\pi(\tau) = \max(0, (\tau - d)s)
\end{equation}

where $d$ represents the deadline of the application and $s$ (slope)
defines the rate the penalty increases as response time increases. For
applications without response-time constraints the deadline can be set
to $0$. Two representative graphs of this type appear in
Figures~\ref{f:pen1} and~\ref{f:pen2}. Penalty functions will be
described further in Section~\ref{app_func}.

\begin{figure}[hb]
\parbox{1.6in}{
\includegraphics*{Penalty1.eps}
\caption{\label{f:pen1}A penalty function with a response time constraint.}
}
\hspace{\fill}
\parbox{1.6in}{
\includegraphics*{Penalty2.eps}
\caption{\label{f:pen2}A penalty function with no response time constraint.}
}
\end{figure}

\subsection{Resource Allocation As Optimization}

\pacora formulates resource allocation as an optimization problem
designed to minimize the total penalty of the system. This approach is
analogous to minimizing user dissatisfaction with the user experience
due to missed deadlines in a client system, and minimizing the contract
penalties paid for violated service-level agreements in a cloud
system.

The optimization selects the allocations for all resources and
resource types at once.  This approach enables the system to make
tradeoffs between resource types.  For example, the system could
choose to allocate more memory bandwidth in lieu of on-chip cache, or
one large core instead of several small cores.  Given that all of the
resources allocated to an application contribute to the response time,
it would difficult to provide predictable response times for
applications considering the allocation of only one resource type at a
time.

A succinct mathematical characterization of this resource allocation scheme is the following:
\begin{eqnarray*}
& \makebox[1in][r]{Minimize}   & \sum_{p\in P} {\pi_p(\tau_p(a_{p,1}\ldots a_{p,n}))}  \\
& \makebox[1in][r]{Subject to} & \sum_{p\in P} a_{p,r} \leq A_r, r = 1,\ldots n        \\
& \makebox[1in][r]{and}        & a_{p,r} \geq 0
\end{eqnarray*}
Here $\pi_p$ is the penalty function for application $p$,
$\tau_p$ is its response time function,
$a_{p,r}$ is the allocation of resource $r$ to application $p$,
and $A_r$ is the total amount of resource $r$ available.  Optimization details are described in Section~\ref{dyn_opt}.

\subsection{Convex Optimization}

If the penalty functions, response time functions, and resource
constraints were arbitrary, little could be done to optimize the total
penalty beyond searching at random for the best allocation.  However,
by framing our resource allocation problem as a \emph{convex
  optimization problem}\cite{BoVa}, two benefits accrue: an optimal
solution will exist without multiple local extrema, and fast,
incremental solutions will become feasible.

A constrained optimization problem is \emph{convex} if both the objective function to be minimized
and the constraint functions that define its feasible solutions are convex functions.
A function $f$ is convex if its domain is a convex set and
$f(\theta x + (1-\theta)y) \leq \theta f(x) + (1-\theta)f(y)$
for all $\theta$ between 0 and 1.
A set is convex if for any two points $x$ and $y$ in the set, the point
$\theta x + (1-\theta)y$
is also in the set for all $\theta$ between 0 and 1.
%If $f$ is differentiable, it is convex if its domain is an open convex set and
%$f(y) \geq f(x) + \nabla f^T\cdot(y-x)$ where $\nabla f$ is the gradient of $f$.
%Put another way, $f$ is convex if its first-order Taylor approximations
%are always global underestimates of its true value.

A convex optimization problem is one that can be expressed in this form:
\begin{eqnarray*}
& \makebox[1in][r]{Minimize}   & f_0(x_1,\ldots x_m)                              \\
& \makebox[1in][r]{Subject to} & f_i(x_1,\ldots f_m) \leq 0, i = 1,\ldots k        \\
& \makebox[1in][r]{where}      & \forall i \quad f_i:\Re^m \rightarrow \Re \mbox{ is convex.}
\end{eqnarray*}


\pacora's resource allocation problem can be transformed into a convex
optimization problem in the $m = |P|\cdot n$ variables $a_{p,r}$ as
long as the penalty functions $\pi_p$ are convex non-decreasing and
the response-time functions $\tau_p$ are convex.  We designed our
functions to meet these constraints, and proofs of their convexity can
be found in Appendix~\ref{convex}. Note that the resource constraints
are affine and therefore convex; they can be rewritten as $\sum_{p\in
  P} (a_{p,r} - A_r) \leq 0$ and $-a_{p,r} \leq 0$.

The convex formulation makes the optimization scale linearly in the
number of resource types and the number of applications.  For client
operating systems with around 100 application running and 10 resource
dimensions, the total number of variables in the optimization problem
is 1000---a very small problem which is solved in microseconds on
current systems.  Cloud systems could have many more than 100
applications running, but the problem size scales linearly and the
potential benefits of a good allocation should scale rapidly with the
size of the system.

\pacora also formulates RTF \emph{creation} as a convex optimization
problem, as explained in Section~\ref{model_creation}.

\subsection{Power and Energy}
In \pacora, we create an artificial application to represent the
interest in reducing system power and energy.  Application 0 is
designated the idle application and receives allocations of all
resources that are left idle, \emph{i.e.,} not allocated to other
applications.  The idle resources can be powered off or put to sleep
if possible to save power.

The ``response time'' for application 0, $\tau_0$, is artificially
defined to be the total system power consumption.
%This response function is affine and monotone nonincreasing in its arguments $a_{0,r}$.
The penalty function $\pi_0$ establishes a system tradeoff between
power and performance that will determine which resources are
allocated to applications to improve performance and which are left
idle.  The penalty function $\pi_0$ can be used to keep total system
power below the parameter $d_0$ to the extent the penalties of other
applications cannot overcome its penalty slope $s_0$. Both $s_0$ and
$d_0$ can be adjusted to reflect the current battery charge in mobile
devices. For example as the battery depletes, $s_0$ could be increased
to force other applications to slow or cease execution.

%Additionally, application 0 functions as \emph{slack} variables in our optimization problem turning the resource bounds into equalities:
%\begin{equation}
%\sum_{p\in P} a_{p,r} - A_r = 0, r = 1,\dots n.
%\end{equation}




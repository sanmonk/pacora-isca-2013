\section{Model Update and Downdate Algorithms}

\subsection{Row Update and Downdate}

Downdating makes an instructive example. A row downdate operation applies
a sequence of Givens rotations to the rows of $Q^T$.
The rotations are calculated to set every $Q^T_{i,dd}$, $i \neq dd$ to zero.
In the end only the diagonal element $Q^T_{dd,dd}$ of column $dd$ will be nonzero.
Since $Q^T$ remains orthogonal, the non-diagonal elements of row $dd$ will also have been zeroed automatically
and the diagonal element will have absolute value 1.
These same rotations are concurrently applied to the elements of $Q^T t$ and to the rows of $R$ $(= Q^T D)$
to reflect the effect that these transformations have on $Q^T$.

It is crucial to select pairs of rows and an order of rotations that preserves the upper triangular structure of $R$
while zeroing all but the diagonal entry of the chosen column $dd$ of $Q^T$.
Since $dd$ is always below the diagonal of $R$ it initially will contain only zeros.
It is therefore sufficient to rotate every non-$dd$ row with row $dd$, proceeding from bottom to top.
The first $m - n - 1$ rotations will keep row $R_{dd,*}$ entirely zero,
and the remaining $n$ rotations will introduce nonzeros in $R_{dd,*}$ from right to left.
The effect on $R$ will be to replace zero elements by nonzero elements only within row $dd$.
At this point, except for a possible difference in overall sign, $R_{dd,*} = D_{dd,*}$.

Now the rows from 0 down through $dd$ of the modified matrices $Q^Tt$ and $R$ and both the rows and columns of the modified $Q^T$
are circularly shifted by one position, moving row $dd$ to the top (and column $dd$ of $Q^T$ to the left edge).
The following is the result:
\begin{displaymath}
\begin{array}{lll}
    \left[\begin{array}{cc}
      \pm1  &  0 \\
      0     &  \tilde{Q}^T
   \end{array}\right]
   \left[\begin{array}{c}
      t_{dd} \\
      \tilde{t}
   \end{array}\right]
   &=&
   \left[\begin{array}{c}
      \pm D_{dd,*} \\
      \tilde{R}
   \end{array}\right] w
   \\
   \\
   &-&
   \left[\begin{array}{cc}
      \pm1  &  0 \\
      0     &  \tilde{Q}^T
   \end{array}\right]
   \left[\begin{array}{c}
      \varepsilon_{dd} \\
      \tilde{\varepsilon}
   \end{array}\right]
\end{array}
\end{displaymath}
The top row has thus been decoupled from the rest of the factorization and may either be deleted or updated with new data.

The update process more or less reverses these steps, adding a new top row to $R$ and $t$ and a row and column to $Q^T$.
Then $R$ is made upper triangular once more by a sequence of Givens rotations that zero its sub-diagonal elements
(formerly the diagonal elements of $\tilde{R}$) one at a time.
These rotations are applied not just to $R$ but also to $Q^Tt$ and of course to $Q^T$ itself.

\subsection{Rank Preservation}

Deciding in advance whether downdating a row of $R$ will reduce its rank
is equivalent to predicting whether one of the Givens rotations, when applied to $R$,
will zero or nearly zero a diagonal entry of $R$.
This is particularly easy to discover because $dd$, the row to be downdated, is initially all zeros in $R$,
\emph{i.e.} in the lower part of the matrix.
In this situation a diagonal entry of $R$, $R_{i,i}$ say, will be compromised if and only if the
cosine of the Givens rotation that involves rows $dd$ and $i$ is nearly zero.
The result will be an interchange of the zero in $R_{dd,i}$ with the nonzero diagonal element $R_{i,i}$.
$R_{dd,i}$ is zero before the rotation because
$R$ was originally upper triangular and prior rotations only involved row subscripts greater than $i$.

\pacora keeps track of the sequence of values in $Q^T_{dd,dd}$ without actually changing $Q^T$
so that if the downdate at location $dd$ is eventually aborted there is nothing to undo.
It is also possible to remember the sines and cosines of the sequence of rotations
so they don't have to be recomputed if success ensues.
A rank-preserving row to downdate will always be available as long as $R$ is sufficiently ``tall''.
Having at least twice as many rows as columns is enough since the number of available rows to downdate
matches or exceeds the maximum possible rank of $R$.

\subsection{Column Update and Downdate}

The active-set NNLS method is based on the idea that since the only constraints are variable positivity
then for all components either the variable or its gradient will be zero at a solution point; see~\cite{BoVa}, page~142.
The active set, denoted by \textbf{Z}, comprises the column subscripts $j$ for which the variable $w_j$ is zero and the gradient $v_j$ is positive.
If a column $j$ not currently in \textbf{Z} happens to acquire a negative $w_j$ after a back-solve, $w_j$ is zeroed,
$j$ is moved into \textbf{Z} and column $j$ is downdated in $R$, thereby making the gradient positive.
Conversely, if a column already in \textbf{Z} happens to acquire a negative gradient $v_j$ it is removed from \textbf{Z} and updated in $R$,
allowing it to further reduce the value of the objective function.

The set \textbf{Z} and its complement \textbf{P} are implemented as an index $idx$
containing a single permutation vector of the column subscripts comprising
the sorted members of \textbf{P} followed by the sorted members of \textbf{Z},
and an offset defining the beginning of \textbf{Z} in the vector.
For example, if columns 1, 3, and 4 are in \textbf{Z} and columns 0, 2, and 5 are in \textbf{P}
then the resulting vector is [0 2 5 1 3 4] and the offset is 3.
Since the offset is just the size of the set \textbf{P} it is naturally called $p$.

Columns are left in place in $R$. The columns of $R$ belonging to \textbf{P} are denoted by $R^p$ and those in \textbf{Z} by $R^z$.
The updating or downdating of a column only involves modifying the index $idx$ to redefine \textbf{P} and \textbf{Z} and then
applying Givens rotations to (all of) the rows of $R$ to restore $R^p$ to upper triangular form.

After initial acquisition of data and $QR$ factorization, each step of \pacora's NNLS algorithm
combines incremental row and column downdates and updates as follows:

\begin{pseudocode}{IncrementalNNLS}{t_0,d_0}
\LOCAL{R,Q^T,Q^Tt,w,v,idx,d,u,done}                              \\
R,Q^T,Q^Tt \GETS \textsc{DndtRow}(R,Q^T,Q^Tt,idx)           \\
R,Q^T,Q^Tt \GETS \textsc{UpdtRow}(t_0,d_0,R,Q^T,Q^Tt,idx)     \\
w \GETS \textsc{BackSolve}(R,Q^Tt,idx)                          \\
v \GETS \textsc{Gradient}(R,Q^Tt,idx)                    \\
\REPEAT
  done \GETS \TRUE                                              \\
  d \GETS \arg\min(w)                                          \\
  \IF w_d < 0 \THEN                                            \\
  \BEGIN
    done \GETS \FALSE                                         \\
    R,Q^T,Q^Tt,idx \GETS \textsc{DndtCol}(R,Q^T,Q^Tt,idx,d)   \\
    w \GETS \textsc{BackSolve}(R,Q^Tt,idx)                    \\
    v \GETS \textsc{Gradient}(R,Q^Tt,idx)              \\
  \END                                                        \\
  u \GETS \arg\min(v)                                         \\
  \IF v_u < 0 \THEN                                           \\
  \BEGIN
    done \GETS \FALSE                                         \\
    R,Q^T,Q^Tt,idx \GETS \textsc{UpdtCol}(R,Q^T,Q^Tt,idx,u)     \\
    w \GETS \textsc{BackSolve}(R,Q^Tt,idx)                    \\
    v \GETS \textsc{Gradient}(R,Q^Tt,idx)              \\
  \END                                                        \\
\UNTIL done                                                   \\
\RETURN{w,v}                                                  \\
\end{pseudocode}

When a column indexed by $d$ in $R^p$ is downdated because $w_d < 0$, that column is moved from \textbf{P} to \textbf{Z} in $idx$.
To restore $R^p$ to upper triangular form, Givens rotations are applied to $R$ at rows $R_{d,*}$ and $R_{k,*}$
where $d < k < p$. The row subscripts $k$ are used in decreasing order from $p-1$ down to $d+1$,
and each rotation zeros the subdiagonal element in $R^p$ of the column indexed by $k$.
As usual, these rotations are also applied to $Q^T$ and $Q^Tt$.
The result in $R^z$ is a ``spike'' of nonzeros in the column that was moved;
it can eventually extend to the bottom of $R$ as \emph{row} updates occur.

Column movements from \textbf{Z} to \textbf{P} are based on the gradient $v$ of the objective function, namely
\begin{eqnarray*}
v &=& 1/2\nabla\|Dw - t\|^2_2 \\
  &=& D^T(Dw - t)             \\
  &=& R^TQ^T(QRw - t)         \\
  &=& R^T(Rw - Q^Tt)          \\
  &=& R^T(-Q^T\varepsilon).
\end{eqnarray*}
If for some column in \textbf{Z} the inner product of the corresponding spiked row in $R^T$ and $-Q^T\varepsilon$ is negative,
the column subscript must be moved to \textsc{P}.
Updating $R^p$ reverses the downdating steps by zeroing the spike via a sequence of Givens rotations on $R$
between adjacent pairs of rows, starting at the bottom and ending at $m,m+1$ where $m$ is the position of the new column in $idx$.
These rotations conveniently extend the columns to the right of $m$ in $R^p$ by one,
thus restoring $R^p$ to upper triangular form. Once again, the rotations are also applied to $Q^T$ and $Q^Tt$.

A new gradient computation and new back-solve for $w$ are clearly necessary after either downdates or updates to columns of $R$.

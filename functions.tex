\section{Application Functions}
%III.	Application Functions
%	a.	Resource Value Functions
%		i.	Structure
%		ii.	Model Building Approach
%			1.	Who measures the data?
%	b.	Penalty Functions
%		i.	Where do the deadlines come from?
%		ii.	How do we set the slopes?
%	c.	Power Model
\subsection*{Penalty Functions}

Penalty functions  are generically defined as members of a family of such functions
so that user preferences for a process $p$ (elided in the discussion below)
can be implemented by assigning values to a few well-understood parameters.
As a process grows or diminishes in importance, its penalty function can be modified accordingly.
In a client operating system,the instantaneous management of penalty function modifications
should be highly automated by the system to avoid unduly burdening the user.
\pacora's penalty functions are non-decreasing piecewise linear functions of the form
$\pi(\tau) = \max(0, s(\tau - d)$.
Two representative graphs of this type appear in Figures~\ref{f:pen1} and~\ref{f:pen2}.

\begin{figure}[hb]
\parbox{1.6in}{
\includegraphics*{Penalty1.eps}
\caption{\label{f:pen1}A penalty function with a response time constraint.}
}
\hspace{\fill}
\parbox{1.6in}{
\includegraphics*{Penalty2.eps}
\caption{\label{f:pen2}A penalty function with no response time constraint.}
}
\end{figure}

The two parameters $d$ and $s$ define the penalty function.
To guarantee it is convex and non-decreasing, $s$ must be non-negative.
The response time $\tau$  is of course non-negative,
and it may be sensible (if not strictly necessary) to convene that $d$ is also.
A response time constrained process has a marked change in slope, namely from 0 to $s$, at the point $\tau= d$.
In the most extreme case $s = \infty$ (implying infinite penalty for the system as a whole when $\tau > d$).  ``Softer'' requirements will doubtless be the rule.
For processes without response time constraints one can set $d = 0$.
This defines linear behavior with $s$ as the rate of penalty increase with response time.

The gradient of process penalty with respect to its resource allocations is useful in controlling the optimization algorithm.
By the chain rule, $\partial\pi/\partial a_r = \partial\pi/\partial\tau\cdot\partial\tau/\partial a_r$.
The first term is well-defined but discontinuous at $\tau = d$ with
$\partial\pi/\partial\tau = \mbox{ if } (\tau - d) \leq 0 \mbox{ then } 0 \mbox{ else } s$.
The problem of estimating the partial derivatives $\partial\tau/\partial a_r$ is dealt with below.

\subsection*{Response Time Functions}

Unlike penalty functions, which describe user preference, a response time function describes process performance based on its resource assignments.
Response time will commonly vary with time as a process changes phase and makes better or worse use of its resources.
To guarantee the objective function is convex, the response time must be also;
this is a plausible requirement akin to the proverbial ``Law of Diminishing Returns''.

Besides the value of the response time function, its gradient or an approximation to it is useful to estimate the relative response time improvement from each type of resource.  A user-level runtime scheduler that schedules work internal to the process may be a good source of data.
Additionally, the resource manager can allocate a modest amount of a resource and measure the change in response time.
Instead of these, \pacora maintains a parameterized analytic response time model with the partial derivatives evaluated from the model \emph{a priori}.

There are examples of response time versus resource behavior that violate convexity.  One such example sometimes occurs in memory allocation, where “plateaus” can sometimes be seen as in Figure~\ref{f:plat}.
\begin{figure}[b]
\parbox{1.6in}{
\includegraphics*{Plateau1.eps}
\caption{\label{f:plat}Response time function with some resource "plateaus".}
}
\hspace{\fill}
\parbox{1.6in}{
\includegraphics*{Plateau2.eps}
\caption{\label{f:plateffect}Net effect of the resource plateaus on the process penalty.}
}
\end{figure}
Such plateaus are typically caused by algorithm adaptations within the process to accommodate variable resource availability.  The response time is really the \emph{minimum} of several convex functions depending on allocation and the point-wise minimum that the process implements fails to preserve convexity.  The effect of the plateaus will be a non-convex penalty as shown in Figure~\ref{f:plateffect} and multiple extrema in the optimization problem will be a likely result.

There are several ways to avoid this problem.  One is based on the observation that such response time functions
will at least be \emph{quasiconvex}.  A function $f$ is quasiconvex if all of its \emph{sublevel sets}
$S_\ell = \{x | f(x) \leq \ell\}$ are convex sets.
Alternatively, $f$ is quasiconvex if its domain is convex and
\begin{displaymath}
f(\theta x + (1-\theta)y) \leq \max(f(x),f(y)), 0 \leq \theta \leq 1
\end{displaymath}

Quasiconvex optimization can be performed by selecting a threshold $\ell$ and replacing the objective function
with a convex constraint function whose sublevel set $S_\ell$ is the same as that of $f$.
Next, one determines whether there is a feasible solution for that particular threshold $\ell$.
Repeated application with a binary search on $\ell$ will reduce the level of feasibility
until the solution is approximated well enough.

Another idea is to use additional constraints to explore convex sub-domains of $\tau$.
For example,the affine constraint $a_{p,r} - \mu \leq 0$ excludes process $p$ from any assignment of resource $r$ exceeding $\mu$.  Similarly, $\mu - a_{p,r} \leq 0$ exludes the opposite possibility.
A binary (or even linear)search of such sub-domains could be used to find the optimal value.

\pacora adopts a simpler idea, modeling response times by functions that are convex by construction
and do not distort response time behavior too much.  This approach is developed more fully below.

\subsection*{Power and Battery Energy}

It is useful to designate a ``process'' to receive allocations of all resources
that are not used elsewhere and are therefore to be powered off if possible.
In \pacora, process 0 plays this role.
The ``response time'' for process 0, $\tau_0$,
is artificially defined to be the total system power consumption.
This response function is affine and monotone nonincreasing in its arguments $a_{0,r}$.

The penalty function $\pi_0$ can now be used to keep total system power below the parameter $d_0$
to the extent the penalties of other processes cannot overcome its penalty slope $s_0$.
Both $s_0$ and $d_0$ can be adjusted to reflect the current battery charge in mobile devices.
As the battery depletes, $\pi_0$ can be used to force other processes to slow or cease execution.
In any event, the slope $s_0$ establishes a system tradeoff between power and performance that
will determine which processor cores are used for each process and which cores are left idle.

This introduction of \emph{slack} resource allocations into Process 0 turns the resource bounds into equalities:
\begin{displaymath}
\sum_{p\epsilon P} a_{p,r} - A_r = 0, r = 1,\dots n.
\end{displaymath}

\subsection*{Response Time Modeling}

While it might be possible to model response times by recording past values and interpolating among them,
this idea has serious shortcomings:
\begin{itemize}
\item The size of the multidimensional response time function tables will be large;
\item Interpolation in many dimensions is computationally expensive;
\item The measurements will be “noisy” and require smoothing;
\item Convexity in the resources may be violated;
\item Gradient estimation will be difficult.
\end{itemize}

An alternative approach is to model the response time functions using parameterized expressions that are convex by construction.
For example, the response time might be modeled as a weighted sum of component terms,
one per bandwidth resource, where each term $w_r/a_r$ is
the amount of work $w_r \geq 0$ divided by $a_r$, the allocation of that bandwidth resource\cite{Snav}.
For example,
one term might model the number of instructions executed divided by total processor MIPS,
another might model storage accesses divided by storage bandwidth allocation and so forth.
Such models will automatically be convex in the allocations because $1/a$ is convex for positive $a$
and because a positively-weighted sum of convex functions is convex.

It is obviously important to guarantee the positivity of the resource allocations.
This can be enforced as the allocations are selected during penalty optimization,
or the response time model can be made to return $\infty$ if any allocation is less than or equal to zero.
This latter idea preserves the convexity of the model and extends its domain to all of $\Re^n$.

Asynchrony and latency tolerance may make response time components overlap partly or fully;
if the latter, then the maximum of the terms might be more appropriate than their sum.
The result will still be convex, though, as will any other norm including the 2-norm,
\emph{i.e.} the square root of the sum of the squares.
This last variation could be viewed as a ``partially overlapped'' compromise between
the 1-norm (sum) describing no overlap and the $\infty$-norm (maximum) describing full overlap.

This scheme also accommodates non-bandwidth resources such as memory,
the general idea being to roughly approximate ``diminishing returns'' in the response time with increasing resources.
For clarity's sake, rather than using $a_r$ indiscriminately for all allocations,
we will denote an allocation of a bandwidth resource by $b_r$ and of a memory resource by $m_r$.

Sometimes a response time component might be better modeled by a term involving a combination of resources.
For example, response time due to memory accesses might be approximated
by a combination of memory bandwidth allocation $b_{r1}$ and cache allocation $m_{r2}$.
Such a model could use the geometric mean of the two allocations in the denominator,
\emph{viz.} $w_{r1,r2}/\sqrt{b_{r1}\cdot m_{r2}}$, without compromising convexity.

This begs the question of how memory affects the response time.
The effect is largely indirect.
Memory permits exploitation of temporal locality and thereby \emph{amplifies} associated bandwidths.
For example, additional main memory may reduce the need for storage or network bandwidth,
and of course increased cache capacity may reduce the need for memory bandwidth.
The effectiveness of cache in reducing bandwidth was studied by
H. T. Kung\cite{Kung}, who developed tight asymptotic bounds on the bandwidth amplification
factor $\alpha(m)$ resulting from a quantity of memory $m$ acting as cache for a variety of computations.
He shows that
\begin{displaymath}
\begin{array}{lll}
\alpha(m) &= \Theta(\sqrt m) & \mbox{for dense linear algebra solvers} \\
          &= \Theta(m^{1/d}) & \mbox{for d-dimensional PDE solvers} \\
          &= \Theta(\log m)  & \mbox{for comparison sorting and FFTs} \\
          &= \Theta(1)       & \mbox{when temporal locality is absent}
\end{array}
\end{displaymath}

For these expressions to make sense, the argument of $\alpha$ should be dimensionless and greater than 1.
Ensuring this might be as simple as letting it be the number of memory resource quanta
(\emph{e.g.} hundreds of memory pages) assigned to the process.
If a process shows diminishing bandwidth amplification as its memory allocation increases, this can be accommodated:
\begin{displaymath}
\alpha(m) = \min(c_1\alpha_1(m),c_2\alpha_2(m)),\;c_1,c_2 \geq 0
\end{displaymath}

Each bandwidth amplification factor might be described by one of the functions above
and included in the denominator of the appropriate component in the response time function model.
For example, the storage response time component for the model of an out-of-core sort process might be
the quantity of storage accesses divided by the product of the storage bandwidth allocation and $\log m$,
the amplification function associated with sorting given a memory allocation of $m$.
Amplification functions for each application might be learned from response time measurements
by observing the effect of varying the associated memory resource while keeping the bandwidth allocation constant.
Alternatively, redundant components, similar except for amplification function, could be included in the model
to let the model fitting process decide among them.

The gradient $\nabla\tau$ is needed by the penalty optimization algorithm.
Since $\tau$ is analytic, generic, and symbolically differentiable
it is a simple matter to compute the gradient of $\tau$ once the model is defined.

\subsection*{Response Time Model Convexity}

We now show that the response time model including the various bandwidth amplification functions is convex
in both the bandwidth and memory resources $b_r$ and $m_r$ given any of the possibilities listed above.
Since norms preserve convexity, this reduces the question to proving each term in the norm is convex.
Since all quantities are positive and both maximum and scaling by a positive constant preserve convexity,
\begin{eqnarray*}
\lefteqn{w/(b\cdot\min(c_1\alpha_1(m),c_2\alpha_2(m)))}   \\
&=& \max(w/(b\cdot c_1\alpha_1(m)),w/(b\cdot c_2\alpha_2(m))).
\end{eqnarray*}
It only remains to show that $1/(b\cdot\alpha(m))$ is convex in $b$ and $m$.

A function is defined to be \emph{log-convex} if its logarithm is convex.
A log-convex function is itself convex because exponentiation preserves convexity,
and the product of log-convex functions is convex because the log of the product is the sum of the logs,
each of which is convex by hypothesis.
Now $1/b$ is log-convex for $b > 0$ because $-\log b$ is convex on that domain.
In a similar way, $\log(1/\sqrt{b\cdot m}) = -(\log b + \log m)/2$
and $\log m^{-1/d} = -(\log m)/d$ are convex.
Finally, $\log (1/\log m)$ is convex because its second derivative is positive for $m > 1$:
\begin{eqnarray*}
\frac{d^2}{dm^2}\log (1/\log m) &=& \frac{d^2}{dm^2}(-\log\log m)  \\
                                  &=& \frac{d}{dm}\left(\frac{-1}{m\log m}\right) \\
                                  &=& \frac{1 + \log m}{(m\log m)^2}.
\end{eqnarray*}

Summing up, a response time function for a process might be modeled by the convex function
\begin{eqnarray*}
\tau(w,b,\alpha,m) &=& \sqrt[p]{\sum_j \left(\frac{w_j}{b_j\cdot\alpha_j(m_j)}\right)^p}  \\
                   &=& \|\mbox{diag} wd^T \|_p
\end{eqnarray*}
where the $w_j$ are the parameters of the model (the “quantities of work”) to be learned,
the components of $d$ satisfy $d_j = 1/(b_j\cdot\alpha_j(m_j))$,
the $b_j$  are the allocations of the bandwidth resources,
the $\alpha_j$ are the bandwidth amplification functions (also to be learned),
the $m_j$ are the allocations of the memory or cache resources that are responsible for the amplifications.
This formulation allows the process response time $\tau$ to be modeled as the $p$-norm of
the component-wise product of a vector $d$ that is computed from the resource allocation
and a learned vector of work quantities $w$.

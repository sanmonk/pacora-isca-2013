\section{Response Time Model Design}\label{rtf_choice}

\subsection*{Alternative Resource Representations}
Asynchrony and latency tolerance may make response time components overlap partly or fully;
if the latter, then the maximum of the terms might be more appropriate than their sum.
The result will still be convex, though, as will any other norm including the 2-norm,
\emph{i.e.} the square root of the sum of the squares.
This last variation could be viewed as a ``partially overlapped'' compromise between
the 1-norm (sum) describing no overlap and the $\infty$-norm (maximum) describing full overlap.

Sometimes a response time component might be better modeled by a term involving a combination of resources.
For example, response time due to memory accesses might be approximated
by a combination of memory bandwidth allocation $b_{r1}$ and cache allocation $m_{r2}$.
Such a model could use the geometric mean of the two allocations in the denominator,
\emph{viz.} $w_{r1,r2}/\sqrt{b_{r1}\cdot m_{r2}}$, without compromising convexity.

This scheme also accommodates non-bandwidth resources such as memory,
the general idea being to roughly approximate ``diminishing returns'' in the response time with increasing resources.
For clarity's sake, rather than using $a_r$ indiscriminately for all allocations,
we will denote an allocation of a bandwidth resource by $b_r$ and of a memory resource by $m_r$.
This begs the question of how memory affects the response time.
The effect is largely indirect.
Memory permits exploitation of temporal locality and thereby \emph{amplifies} associated bandwidths.
For example, additional main memory may reduce the need for storage or network bandwidth,
and of course increased cache capacity may reduce the need for memory bandwidth.
The effectiveness of cache in reducing bandwidth was studied by
H. T. Kung\cite{Kung}, who developed tight asymptotic bounds on the bandwidth amplification
factor $\alpha(m)$ resulting from a quantity of memory $m$ acting as cache for a variety of computations.
He shows that
\begin{displaymath}
\begin{array}{lll}
\alpha(m) &= \Theta(\sqrt m) & \mbox{for dense linear algebra solvers} \\
          &= \Theta(m^{1/d}) & \mbox{for d-dimensional PDE solvers} \\
          &= \Theta(\log m)  & \mbox{for comparison sorting and FFTs} \\
          &= \Theta(1)       & \mbox{when temporal locality is absent}
\end{array}
\end{displaymath}

For these expressions to make sense, the argument of $\alpha$ should be dimensionless and greater than 1.
Ensuring this might be as simple as letting it be the number of memory resource quanta
(\emph{e.g.} hundreds of memory pages) assigned to the application.
If a application shows diminishing bandwidth amplification as its memory allocation increases, this can be accommodated:
\begin{displaymath}
\alpha(m) = \min(c_1\alpha_1(m),c_2\alpha_2(m)),\;c_1,c_2 \geq 0
\end{displaymath}

Each bandwidth amplification factor might be described by one of the functions above
and included in the denominator of the appropriate component in the response time function model.
For example, the storage response time component for the model of an out-of-core sort application might be
the quantity of storage accesses divided by the product of the storage bandwidth allocation and $\log m$,
the amplification function associated with sorting given a memory allocation of $m$.
Amplification functions for each application might be learned from response time measurements
by observing the effect of varying the associated memory resource while keeping the bandwidth allocation constant.
Alternatively, redundant components, similar except for amplification function, could be included in the model
to let the model fitting process decide among them.

\subsection*{Stochastic Models}

\subsection*{Initial Model Evaluation}




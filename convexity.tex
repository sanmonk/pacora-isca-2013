\section{Problem Convexity}

\subsection*{Response Time Model Convexity}

We now show that the response time model including the various bandwidth amplification functions is convex
in both the bandwidth and memory resources $b_r$ and $m_r$ given any of the possibilities listed above.
Since norms preserve convexity, this reduces the question to proving each term in the norm is convex.
Since all quantities are positive and both maximum and scaling by a positive constant preserve convexity,
\begin{eqnarray*}
\lefteqn{w/(b\cdot\min(c_1\alpha_1(m),c_2\alpha_2(m)))}   \\
&=& \max(w/(b\cdot c_1\alpha_1(m)),w/(b\cdot c_2\alpha_2(m))).
\end{eqnarray*}
It only remains to show that $1/(b\cdot\alpha(m))$ is convex in $b$ and $m$.

A function is defined to be \emph{log-convex} if its logarithm is convex.
A log-convex function is itself convex because exponentiation preserves convexity,
and the product of log-convex functions is convex because the log of the product is the sum of the logs,
each of which is convex by hypothesis.
Now $1/b$ is log-convex for $b > 0$ because $-\log b$ is convex on that domain.
In a similar way, $\log(1/\sqrt{b\cdot m}) = -(\log b + \log m)/2$
and $\log m^{-1/d} = -(\log m)/d$ are convex.
Finally, $\log (1/\log m)$ is convex because its second derivative is positive for $m > 1$:
\begin{eqnarray*}
\frac{d^2}{dm^2}\log (1/\log m) &=& \frac{d^2}{dm^2}(-\log\log m)  \\
                                  &=& \frac{d}{dm}\left(\frac{-1}{m\log m}\right) \\
                                  &=& \frac{1 + \log m}{(m\log m)^2}.
\end{eqnarray*}

Summing up, a response time function for a process might be modeled by the convex function
\begin{eqnarray*}
\tau(w,b,\alpha,m) &=& \sqrt[p]{\sum_j \left(\frac{w_j}{b_j\cdot\alpha_j(m_j)}\right)^p}  \\
                   &=& \|\mbox{diag} wd^T \|_p
\end{eqnarray*}
where the $w_j$ are the parameters of the model (the “quantities of work”) to be learned,
the components of $d$ satisfy $d_j = 1/(b_j\cdot\alpha_j(m_j))$,
the $b_j$  are the allocations of the bandwidth resources,
the $\alpha_j$ are the bandwidth amplification functions (also to be learned),
the $m_j$ are the allocations of the memory or cache resources that are responsible for the amplifications.
This formulation allows the process response time $\tau$ to be modeled as the $p$-norm of
the component-wise product of a vector $d$ that is computed from the resource allocation
and a learned vector of work quantities $w$.